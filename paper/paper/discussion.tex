

\section{Discussion}
The curve of \%percent of reported counts over revisions indicates that specific backfill patterns exist. Quantification of such backfill patterns and the evaluation of projection results remains a challenge. We use weighted interval scores to evaluate our log-scale projection to investigate the relative errors which can vary broadly in different regions and time periods.  The current strategy is reasonable enough but not perfect. As shown in Figure X shows examples of poor projection performance for the locations with small target fractions. Minor absolute errors are magnified in the relative error evaluation system. We believe that he same relative error is less acceptable for areas with higher infection densities. 

%% TODO: Not the latest version.  
%% Use log 10 or natural log scale? natural log add the second axis? 
\begin{figure}
    \centering
    \includegraphics[width=\textwidth]{figs/evl_comparison_states.png}
    \caption{\textit{red lines represent the 10 states with the worst prediction performance while blue lines represent the 10 states with the best prediction performance}}
\end{figure}

Due to the exponential nature, situations in which models missed the beginning of upswings are more strongly emphasized while failing to predict a downturn following a peak is less severely penalized. 
\begin{figure}
    \centering
    \includegraphics[width=\textwidth]{figs/pred_problem_in_ca.pdf}
    \caption{\textit{the backfill correction}}
\end{figure}

\begin{figure}
    \centering
    \includegraphics[width=\textwidth]{figs/new_covs_zoom_in.pdf}
    \caption{\textit{the backfill correction}}
\end{figure}