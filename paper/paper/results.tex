
\section{Evaluation}

We now evaluate projection performance for all locations of interest (50 states and top 200 populous counties)  and for all test dates between 2021-08-30 and 2023-03-11 inclusive (589 days in total) against the target values as defined in section 2.1. To reiterate, in our real-time backfill correction experiment, we restrict our access to data that would have been available at each test date $S$. In order to save time computationally, we train the model for every 14 consecutive days but not daily. 

We use the weighted interval score(WIS) to evaluate the distance between the projected distribution and the target value. Suppose our projected values are notated as $\hat{Q}_{(Y_{itl})}(\tau)$ after exponentiation, which represents the projected $\tau$th quantile of $Y_{itL}$ given $y_{itl}$ and corresponding covariates. Notice that in the count projection, for each location $i$ and reference date $t$ pair, we have multiple data points with different lags to project the same $Y_{itL}$. To distinguish the projection results from these points, we will re-write them as $\hat{Y}_{itl}|X_{itl}$. Therefore, the WIS scores are calculated by comparing the distribution of $\hat{Y}_{itl}$ on log scale with the distribution $Y_{itL}$ on log scale no matter for counts or fractions. 

Figure X shows an example of count projection based on CHNG outpatient COVID claims in California. The real-time report, which will be revised in the next two months, largely deviates from the truth temporarily. In such cases, our model can avoid misjudgment of trends due to claims counts that are not reported in time. The distribution of fractions is much different from the previous one, and the pattern is more complicated since the revisions of the fractions are no longer monotonically increasing. 
%% Use projection for 7dav instead of daily
Figure X shows the example of fraction projection based on CHNG outpatient COVID-19 fractions in Harris County, TX. With only a limited number of revisions, the predictions provided by our model are close enough to the target values, which are available two months later after the first release. 

\begin{figure}
    \centering
    \includegraphics[width=\textwidth]{}
    \caption{\textit{}}
\end{figure}

To have a broad view of the projection performance, the projection errors are aggregated into an average on log scale and can be interpreted as the geometric average of relative error if exponentiated. The real-time observation can roughly serve as the projected median from a baseline model. For the purposes of making fair comparisons, we include the mean absolute error on log scale using the projected media




Test on other datasets:
choice 1) MA DPH : well known, but not available for all dates of interest (starting from 2021, 1, 4. End of 2022, 7, 8)
choice 2) Quidel Antigen Data
choice 3) CHNG Inpatient Data


